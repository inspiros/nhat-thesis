%!TEX root = ../main.tex

%\renewcommand{\abstractname}{Kurzfassung}
\begin{abstract}
The research field of Music Information Retrieval is concerned with the automatic analysis of musical characteristics. One aspect that has not received much attention so far is the automatic analysis of sung lyrics. On the other hand, the field of Automatic Speech Recognition has produced many methods for the automatic analysis of speech, but those have rarely been employed for singing. This thesis analyzes the feasibility of applying various speech recognition methods to singing, and suggests adaptations. In addition, the routes to practical applications for these systems are described. Five tasks are considered: Phoneme recognition, language identification, keyword spotting, lyrics-to-audio alignment, and retrieval of lyrics from sung queries.\\
The main bottleneck in almost all of these tasks lies in the recognition of phonemes from sung audio. Conventional models trained on speech do not perform well when applied to singing. Training models on singing is difficult due to a lack of annotated data. This thesis offers two approaches for generating such data sets. For the first one, speech recordings are made more ``song-like''. In the second approach, textual lyrics are automatically aligned to an existing singing data set. In both cases, these new data sets are then used for training new acoustic models, offering considerable improvements over models trained on speech.\\
Building on these improved acoustic models, speech recognition algorithms for the individual tasks were adapted to singing by either improving their robustness to the differing characteristics of singing, or by exploiting the specific features of singing performances. Examples of improving robustness include the use of keyword-filler HMMs for keyword spotting, an i-vector approach for language identification, and a method for alignment and lyrics retrieval that allows highly varying durations. Features of singing are utilized in various ways: In an approach for language identification that is well-suited for long recordings; in a method for keyword spotting based on phoneme durations in singing; and in an algorithm for alignment and retrieval that exploits known phoneme confusions in singing.\\
\end{abstract}


% \begin{otherlanguage}{ngerman}
% %\renewcommand{\abstractname}{Kurzfassung}
% %\newcommand{\dtabstract}{\hyphenpenalty=10000}
% %{\dtabstract
% \begin{abstract}
% Das Gebiet des Music Information Retrieval befasst sich mit der automatischen Analyse von musikalischen Charakteristika. Ein Aspekt, der bisher kaum erforscht wurde, ist dabei der gesungene Text. Auf der anderen Seite werden in der automatischen Spracherkennung viele Methoden für die automatische Analyse von Sprache entwickelt, jedoch selten für Gesang. Die vorliegende Arbeit untersucht die Anwendung von Methoden aus der Spracherkennung auf Gesang und beschreibt mögliche Anpassungen. Zudem werden Wege zur praktischen Anwendung dieser Ansätze aufgezeigt. Fünf Themen werden dabei betrachtet: Phonemerkennung, Sprachenidentifikation, Schlagwortsuche, Text-zu-Gesangs-Alignment und Suche von Texten anhand von gesungenen Anfragen.\\
% Das größte Hindernis bei fast allen dieser Themen ist die Erkennung von Phonemen aus Gesangsaufnahmen. Herkömmliche, auf Sprache trainierte Modelle, bieten keine guten Ergebnisse für Gesang. Das Trainieren von Modellen auf Gesang ist schwierig, da kaum annotierte Daten verfügbar sind. Diese Arbeit zeigt zwei Ansätze auf, um solche Daten zu generieren. Für den ersten wurden Sprachaufnahmen künstlich gesangsähnlicher gemacht. Für den zweiten wurden Texte automatisch zu einem vorhandenen Gesangsdatensatz zugeordnet. Die neuen Datensätze wurden zum Trainieren neuer Modelle genutzt, welche deutliche Verbesserungen gegenüber sprachbasierten Modellen bieten.\\
% Auf diesen verbesserten akustischen Modellen aufbauend wurden Algorithmen aus der Spracherkennung für die verschiedenen Aufgaben angepasst, entweder durch das Verbessern der Robustheit gegenüber Gesangscharakteristika oder durch das Ausnutzen von hilfreichen Besonderheiten von Gesang. Beispiele für die verbesserte Robustheit sind der Einsatz von Keyword-Filler-HMMs für die Schlagwortsuche, ein i-Vector-Ansatz für die Sprachenidentifikation sowie eine Methode für das Alignment und die Textsuche, die stark schwankende Phonemdauern nicht bestraft. Die Besonderheiten von Gesang werden auf verschiedene Weisen genutzt: So z.B. in einem Ansatz für die Sprachenidentifikation, der lange Aufnahmen benötigt; in einer Methode für die Schlagwortsuche, die bekannte Phonemdauern in Gesang mit einbezieht; und in einem Algorithmus für das Alignment und die Textsuche, der bekannte Phonemkonfusionen verwertet.
% \end{abstract}
% \end{otherlanguage}
